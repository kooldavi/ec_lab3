\section{Bericht}

\subsection{Einleitung}
In diesem Versuch haben wir vor, die in Lab 2 errechneten Werte für den PID-Regler in den vorhandenen Code einzufügen und so ein funktionierendes Experiment zu haben. Zusätzlich mussten wir die Werte für $MIDPOS$ und $DELTA$ berechnen und den Willkommenstext auf der LCD-Anzeige ändern.

\subsubsection{Verwendung von OCR3A}
OCR3A ist das Timer Output Compare Register, ...\\
\textcolor{red}{TODO}

\subsubsection{Berechnung von MIDPOS}
Der richtige MIDPOS Wert gibt an, wann die Fläche vollkommen horizontal steht. Zur Berechnung von MIDPOS benutzten wir folgende Formel: \\
\begin{equation}
MIDPOS  = Frequenz / Prescaler / 1000 * Pulslaenge
\end{equation}
Die Frequenz war vorgegeben mit $16 Mhz$. Der Prescaler-Wert wurde am Port TCCR3B mit $64$ festgelegt und konnte mit Hilfe des Datenblattes aus dem Code abgelesen werden. Die Pulslänge bei horizontaler Lage wurde in der Aufgabenstellung erwähnt und lag bei $1,5 ms$.\\
Das Ergebnis war $MIDPOS = 375$, allerdings war die Fläche nicht vollständig horizontal, nach kurzem Anpassen kamen wir auf den Wert $MIDPOS = 355$.

\subsubsection{Berechnung von DELTA}
DELTA ist die Differenz zwischen der maximalen linken oder rechten Position der Fläche und der mittleren Position (MIDPOS). Für die Berechnung des Wertes von DELTA verwendeten wir die folgende Formel:
\begin{equation}
DELTA = (Frequenz / Prescaler / 1000 * Pulslaenge) - MIDPOS
\end{equation}
Für den Wert der Pulslänge mussten wir diesmal allerdings $2 ms$ verwenden da wir den Wert für die maximale rechte Position der Fläche benötigten. Von diesem Wert zogen wir $MIDPOS$ ab um die Differenz zu erhalten. Nun mussten wir wieder den Wert leicht anpassen und kamen auf das Ergebnis $Delta = 480$.

\subsubsection{Willkommens-Anzeige}
Die Willkommensnachricht auf der LCD Anzeige konnten wir in der Datei $HMI.c$ in Zeile 260 einfach ändern.

\subsubsection{PID Implementation}
Die Implementierung des PID-Reglers erfolgte komplett in der Datei $PID.c$. Die Formel die wir verwendeten für den Wert des PID-Reglers lautete:
\begin{multline}
y = (proportional * error) + (integral * errorSum * sampleTime) + \\
(derivative * (error * errorLast) / sampleTime)
\end{multline}
Die Werte für $proportional$, $integral$ und $derivative$ konnten wegen Abweichung im realen Versuch leider nicht übernommen werden. Nach längerem kalibrieren kamen wir auf die Werte $proportional = 0,19$, $integral = 0,018$ und $derivative = 0,51$.\\
Als $sampleTime$ verwendeten wir die in der Aufgabenstellung erwähnte Signalperiode von $18 ms$ ($0,018 s$). $error$ berechneten wir als Differenz zwischen der Zielposition des Balls und der aktuellen Position.

\subsubsection{Vergleich der Ergebnisse mit Versuch 2}
Im Vergleich zu Versuch 2 benötigten wir beim eigentlichen Modell vollkommen andere PID-Werte. Ein möglicher Grund dafür ist, dass wir in Versuch 2 ein kontinuierliches System simuliert haben, in Versuch 3 aber mit einem diskreten System gearbeitet haben.
\textcolor{red}{TODO}

\subsubsection{Queues und Tasks}
Es gibt die drei Queues QueueTaster, QueueSensor und QueueServo sowie die sechs Tasks vSensor, vIO\_SRAM\_to\_LCD, vTaster, vHMI, vServo und vPID.\\
\\
{\bfseries QueueTaster}\\
Diese Queue erhält Input-Daten von $vTaster$, diese werden dann weiter gegeben an $vHMI$ und an $change\_para$.\\
{\bfseries QueueSensor} \\
$QueueSensor$ empfängt die Positionsdaten des Balls von $vSensor$ und sendet diese an $vPID$.\\
{\bfseries QueueServo} \\
$QueueServo$ verwaltet den PID-Wert der in $vPID$ berechnet wurde und gibt sie an $vServo$ weiter.\\
{\bfseries vSensor} \\
Dieser Task berechnet die aktuelle Ballposition mit Hilfe der Sensoren und sendet die Position an $QueueSensor$.\\
{\bfseries vIO\_SRAM\_to\_LCD} \\
Dieser Task kopiert Daten vom RAM zum LCD und positioniert den Cursor im Menü.\\
{\bfseries vTaster} \\
$vTaster$ liest Input Daten an den Pins und sendet diese Parameter an $QueueTaster$.\\
{\bfseries vHMI} \\
Dieser Task steuert/verwaltet die Menüs und Funktionen, welche auf dem LCD dargestellt werden, überprüft welcher Parameter zur Zeit in $Queuetaster$ ist und verwendet diesen zur Menüsteuerung.\\
{\bfseries vServo} \\
Dieser Task berechnet den Vergleichswert des Servosignals und setzt diesen Wert auf den Port $OCR3A$. $vServo$ erhält den Wert $Beta$ der für die Berechnung benötigt wird aus $QueueServo$.\\
{\bfseries vPID} \\
Dieser Task holt sich die aktuelle Ballposition aus $QueueSensor$ und berechnet damit den aktuellen PID-Wert.\\


\subsubsection{Inter Task Kommunikation}
\textcolor{red}{TODO}

\subsection{Ergebnis}
Mit unseren aktualisierten PID-Werten ist es uns gelungen den Ball in weniger als zehn Sekunden an der gewünschten Stelle stehen bleiben zu lassen.\\

\subsection{Fazit}
\textcolor{red}{TODO}